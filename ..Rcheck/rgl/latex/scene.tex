\HeaderA{scene}{scene management}{scene}
\aliasA{clear3d}{scene}{clear3d}
\aliasA{pop3d}{scene}{pop3d}
\aliasA{rgl.clear}{scene}{rgl.clear}
\aliasA{rgl.ids}{scene}{rgl.ids}
\aliasA{rgl.pop}{scene}{rgl.pop}
\keyword{dynamic}{scene}
\begin{Description}\relax
Clear shapes, lights, bbox
\end{Description}
\begin{Usage}
\begin{verbatim}
clear3d( type = c("shapes", "bboxdeco") ) 
rgl.clear( type = "shapes" )
pop3d( ... )
rgl.pop( type = "shapes", id = 0 )  
rgl.ids( type = "shapes" )
\end{verbatim}
\end{Usage}
\begin{Arguments}
\begin{ldescription}
\item[\code{type}] Select subtype(s):
\describe{
\item["shapes"] shape stack
\item["lights"] light stack
\item["bboxdeco"] bounding box
\item["viewpoint"] viewpoint
\item["all"] all of the above
}

\item[\code{id}] vector of ID numbers of items to remove
\item[\code{...}] generic arguments passed through to RGL-specific (or other) functions
\end{ldescription}
\end{Arguments}
\begin{Details}\relax
RGL holds two stacks. One is for shapes and the other is for lights. 
\code{clear3d} and \code{rgl.clear} clear the specified stack, or restore
the defaults for the bounding box (not visible) or viewpoint.  By default
with \code{id=0} \code{rgl.pop} removes 
the top-most (last added) node on the shape stack.  The \code{id} argument
may be used to specify arbitrary item(s) to remove from the specified stack.

\code{rgl.ids} returns a dataframe containing the IDs in the currently active
rgl window, along with an indicator of their type.

For convenience, \code{type="shapes"} and 
\code{id=1} signifies the bounding box.

Note that clearing the light stack leaves the scene in darkness; it should normally
be followed by a call to \code{\LinkA{rgl.light}{rgl.light}} or \code{\LinkA{light3d}{light3d}}.
\end{Details}
\begin{SeeAlso}\relax
\code{\LinkA{rgl}{rgl}}
\code{\LinkA{rgl.bbox}{rgl.bbox}}
\code{\LinkA{rgl.light}{rgl.light}}
\end{SeeAlso}
\begin{Examples}
\begin{ExampleCode}
  x <- rnorm(100)
  y <- rnorm(100)
  z <- rnorm(100)
  p <- plot3d(x, y, z, type='s')
  rgl.ids()
  lines3d(x, y, z)
  rgl.ids()
  if (interactive()) {
    readline("Hit enter to change spheres")
    rgl.pop(id = p[c("data", "box.lines")])
    spheres3d(x, y, z, col="red", radius=1/5)
    box3d()
  }
\end{ExampleCode}
\end{Examples}

