\HeaderA{subdivision3d}{generic subdivision surface method}{subdivision3d}
\aliasA{deform.qmesh3d}{subdivision3d}{deform.qmesh3d}
\aliasA{divide.qmesh3d}{subdivision3d}{divide.qmesh3d}
\aliasA{normalize.qmesh3d}{subdivision3d}{normalize.qmesh3d}
\methaliasA{subdivision3d.qmesh3d}{subdivision3d}{subdivision3d.qmesh3d}
\keyword{dynamic}{subdivision3d}
\begin{Description}\relax
The Subdivision surface algorithm divide and refine (deform) a
given mesh recursively to certain degree (depth).
The qmesh3d algorithm consists of two stages: divide and deform.
The divide step generates for each quad four new quads, the deform
step drags the points (refinement step).
\end{Description}
\begin{Usage}
\begin{verbatim}
  subdivision3d( x,  ...)
  subdivision3d.qmesh3d( x, depth=1, normalize=FALSE, deform=TRUE, ... )
  divide.qmesh3d(mesh,  vb=mesh$vb, ib=mesh$ib )
  normalize.qmesh3d(mesh)
  deform.qmesh3d(mesh,vb=mesh$vb,ib=mesh$ib)
\end{verbatim}
\end{Usage}
\begin{Arguments}
\begin{ldescription}
\item[\code{x}] 3d geometry mesh
\item[\code{mesh}] 3d geometry mesh
\item[\code{depth}] recursion depth
\item[\code{normalize}] normalize qmesh3d coordinates after division if deform is TRUE otherwise
\item[\code{deform}] deform mesh after normalization
\item[\code{ib}] vector of indices (i.e., which sets of vertices to combine
into quads); 4xn matrix or equivalent vector
\item[\code{vb}] vector of vertices: 4xn matrix (rows x,y,z,h) or equivalent
vector, where h indicates scaling of each plotted quad
\item[\code{...}] other arguments (unused)
\end{ldescription}
\end{Arguments}
\begin{Details}\relax
Generic subdivision surface method. Currently there exists
an algorithm that can be applied on qmesh3d objects.
\end{Details}
\begin{SeeAlso}\relax
\code{\LinkA{r3d}{r3d}}
\code{\LinkA{qmesh3d}{qmesh3d}}
\end{SeeAlso}
\begin{Examples}
\begin{ExampleCode}
  open3d()
  shade3d( subdivision3d( cube3d(), depth=3 ), color="red", alpha=0.5 )
\end{ExampleCode}
\end{Examples}

