\HeaderA{rgl.snapshot}{export screenshot}{rgl.snapshot}
\aliasA{snapshot3d}{rgl.snapshot}{snapshot3d}
\keyword{dynamic}{rgl.snapshot}
\begin{Description}\relax
Saves the screenshot as png file.
\end{Description}
\begin{Usage}
\begin{verbatim}
rgl.snapshot( filename, fmt="png" )
snapshot3d( ... )
\end{verbatim}
\end{Usage}
\begin{Arguments}
\begin{ldescription}
\item[\code{filename}] full path to filename.
\item[\code{fmt}] image export format, currently supported: png 
\item[\code{...}] arguments to pass to \code{rgl.snapshot} 
\end{ldescription}
\end{Arguments}
\begin{Details}\relax
Animations can be created in a loop modifying the scene and saving 
each screenshot to a file. Various graphics programs (e.g. ImageMagick)
can put these together into a single animation. (See example below)
\end{Details}
\begin{SeeAlso}\relax
\code{\LinkA{rgl.viewpoint}{rgl.viewpoint}}
\end{SeeAlso}
\begin{Examples}
\begin{ExampleCode}

## Not run: 

#
# create animation
#

shade3d(oh3d(), color="red")
rgl.viewpoint(0,20)

setwd(tempdir())
for (i in 1:45) {
  rgl.viewpoint(i,20)
  filename <- paste("pic",formatC(i,digits=1,flag="0"),".png",sep="")
  rgl.snapshot(filename)
}
## Now run ImageMagick command:
##    convert -delay 10 *.png -loop 0 pic.gif
## End(Not run)

\end{ExampleCode}
\end{Examples}

