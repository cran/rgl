\HeaderA{light}{add light source}{light}
\aliasA{light3d}{light}{light3d}
\aliasA{rgl.light}{light}{rgl.light}
\keyword{dynamic}{light}
\begin{Description}\relax
add a light source to the scene.
\end{Description}
\begin{Usage}
\begin{verbatim}
light3d(theta = 0, phi = 15, ...)
rgl.light( theta = 0, phi = 0, viewpoint.rel = TRUE, ambient = "#FFFFFF", 
           diffuse = "#FFFFFF", specular = "#FFFFFF")
\end{verbatim}
\end{Usage}
\begin{Arguments}
\begin{ldescription}
\item[\code{theta, phi}] polar coordinates
\item[\code{viewpoint.rel}] logical, if TRUE light is a viewpoint light that is positioned relative to the current viewpoint
\item[\code{ambient, diffuse, specular }] 
\item[\code{...}] generic arguments passed through to RGL-specific (or other) functions
\end{ldescription}
\end{Arguments}
\begin{Details}\relax
Up to 8 light sources are supported. They are positioned either in world space
or relative to the camera using polar coordinates. Light sources are directional.
\end{Details}
\begin{Value}
This function is called for the side effect of adding a light.  A light ID is
returned to allow \code{\LinkA{rgl.pop}{rgl.pop}} to remove it.
\end{Value}
\begin{SeeAlso}\relax
\code{\LinkA{rgl.clear}{rgl.clear}}
\code{\LinkA{rgl.pop}{rgl.pop}}
\end{SeeAlso}

