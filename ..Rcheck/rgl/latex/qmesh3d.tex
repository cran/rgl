\HeaderA{qmesh3d}{3D Quadrangle Mesh objects}{qmesh3d}
\aliasA{cube3d}{qmesh3d}{cube3d}
\aliasA{dot3d}{qmesh3d}{dot3d}
\methaliasA{dot3d.qmesh3d}{qmesh3d}{dot3d.qmesh3d}
\aliasA{oh3d}{qmesh3d}{oh3d}
\aliasA{shade3d}{qmesh3d}{shade3d}
\methaliasA{shade3d.qmesh3d}{qmesh3d}{shade3d.qmesh3d}
\aliasA{wire3d}{qmesh3d}{wire3d}
\methaliasA{wire3d.qmesh3d}{qmesh3d}{wire3d.qmesh3d}
\keyword{dynamic}{qmesh3d}
\begin{Description}\relax
3D Quadrangle Mesh object creation and a collection of sample objects.
\end{Description}
\begin{Usage}
\begin{verbatim}
  qmesh3d(vertices, indices, homogeneous = TRUE, material = NULL)
  cube3d(trans = identityMatrix(), ...)  # cube object
  oh3d(trans = identityMatrix(), ...)    # an 'o' object
  
  dot3d(x, ...)   # draw dots at the vertices of an object
  ## S3 method for class 'qmesh3d':
  dot3d(x, override = TRUE, ...)
  wire3d(x, ...)  # draw a wireframe object
  ## S3 method for class 'qmesh3d':
  wire3d(x, override = TRUE, ...)
  shade3d(x, ...) # draw a shaded object
  ## S3 method for class 'qmesh3d':
  shade3d(x, override = TRUE, ...)
\end{verbatim}
\end{Usage}
\begin{Arguments}
\begin{ldescription}
\item[\code{x}] a qmesh3d object (class qmesh3d)
\item[\code{vertices}] 3- or 4-component vector of coordinates
\item[\code{indices}] 4-component vector of quad indices
\item[\code{homogeneous}] logical indicating if homogeneous (four component) coordinates are used.
\item[\code{material}] material properties for later rendering
\item[\code{trans}] transformation to apply to objects; see below for defaults
\item[\code{...}] additional rendering parameters
\item[\code{override}] should the parameters specified here override those stored in the object?
\end{ldescription}
\end{Arguments}
\begin{Details}\relax
The \code{cube3d} and \code{oh3d} objects optionally take a matrix transformation as 
an argument.  This transformation is applied to all vertices of the default shape.

The default is an identity transformation.  Use \code{par3d("userMatrix")} to render the
object vertically in the current user view.
\end{Details}
\begin{Value}
\code{qmesh3d}, \code{cube3d}, and \code{oh3d} return \code{qmesh3d} objects.

\code{dot3d}, \code{wire3d}, and \code{shade3d} are called for their side effect
of drawing an object into the scene; they return an object ID.
\end{Value}
\begin{SeeAlso}\relax
\code{\LinkA{r3d}{r3d}}, \code{\LinkA{par3d}{par3d}}
\end{SeeAlso}
\begin{Examples}
\begin{ExampleCode}

  # generate a quad mesh object

  vertices <- c( 
     -1.0, -1.0, 0, 1.0,
      1.0, -1.0, 0, 1.0,
      1.0,  1.0, 0, 1.0,
     -1.0,  1.0, 0, 1.0
  )
  indices <- c( 1, 2, 3, 4 )
  
  open3d()  
  wire3d( qmesh3d(vertices,indices) )
  
  # render 4 meshes vertically in the current view

  open3d()  
  bg3d("gray")
  l0 <- oh3d(tran = par3d("userMatrix"), color = "green" )
  shade3d( translate3d( l0, -6, 0, 0 ))
  l1 <- subdivision3d( l0 )
  shade3d( translate3d( l1 , -2, 0, 0 ), color="red", override = FALSE )
  l2 <- subdivision3d( l1 )
  shade3d( translate3d( l2 , 2, 0, 0 ), color="red", override = TRUE )
  l3 <- subdivision3d( l2 )
  shade3d( translate3d( l3 , 6, 0, 0 ), color="red" )
  
\end{ExampleCode}
\end{Examples}

