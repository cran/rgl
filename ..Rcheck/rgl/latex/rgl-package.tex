\HeaderA{rgl-package}{3D visualization device system}{rgl.Rdash.package}
\aliasA{rgl}{rgl-package}{rgl}
\aliasA{rgl.close}{rgl-package}{rgl.close}
\aliasA{rgl.cur}{rgl-package}{rgl.cur}
\aliasA{rgl.open}{rgl-package}{rgl.open}
\aliasA{rgl.quit}{rgl-package}{rgl.quit}
\aliasA{rgl.set}{rgl-package}{rgl.set}
\keyword{dynamic}{rgl-package}
\begin{Description}\relax
3D real-time rendering device-driver system
\end{Description}
\begin{Usage}
\begin{verbatim}
rgl.open()     # open new device
rgl.close()    # close current device
rgl.cur()      # returns active device ID
rgl.set(which) # set device as active
rgl.quit()     # shutdown rgl device system
\end{verbatim}
\end{Usage}
\begin{Arguments}
\begin{ldescription}
\item[\code{which}] device ID
\end{ldescription}
\end{Arguments}
\begin{Details}\relax
RGL is a 3D real-time rendering device driver system for R.
Multiple devices are managed at a time, where one has the current focus
that receives instructions from the R command-line.
The device design is oriented towards the R device metaphor. If you send
scene management instructions, and there's no device open, it will be opened
automatically.
Opened devices automatically get the current device focus. The focus may be
changed by using rgl.set().
rgl.quit() shuts down the rgl subsystem and all open devices, 
detaches the package including the shared library and additional system libraries.
\end{Details}
\begin{SeeAlso}\relax
\code{\LinkA{rgl.clear}{rgl.clear}}, 
\code{\LinkA{rgl.pop}{rgl.pop}},
\code{\LinkA{rgl.viewpoint}{rgl.viewpoint}},
\code{\LinkA{rgl.light}{rgl.light}},
\code{\LinkA{rgl.bg}{rgl.bg}},
\code{\LinkA{rgl.bbox}{rgl.bbox}},
\code{\LinkA{rgl.points}{rgl.points}},
\code{\LinkA{rgl.lines}{rgl.lines}},
\code{\LinkA{rgl.triangles}{rgl.triangles}},
\code{\LinkA{rgl.quads}{rgl.quads}},
\code{\LinkA{rgl.texts}{rgl.texts}},
\code{\LinkA{rgl.surface}{rgl.surface}},
\code{\LinkA{rgl.spheres}{rgl.spheres}},
\code{\LinkA{rgl.sprites}{rgl.sprites}},
\code{\LinkA{rgl.snapshot}{rgl.snapshot}}
\end{SeeAlso}
\begin{Examples}
\begin{ExampleCode}
example(surface3d)
example(plot3d)
\end{ExampleCode}
\end{Examples}

